\section{Proofs and Mathematical Details}

\subsection{Regarding Signature Share NIZK Proofs}
\label{sec:proof-sigshare}

In Shoup's original scheme~\cite{shoup-sig}, proofs of signature share correctness are utilized to determine
cheating members of the root collective. We sketch an argument that
Shoup's threshold RSA signature scheme~\cite{shoup-sig} remains
secure, even in the absence of these proofs of correctness, insofar as
a corrupted or malicious player cannot forge a signature.

First, note that by stripping the proofs of correct construction from
the signature shares, we are not adding new information or changing
the way in which information is conveyed. The behavior of the honest
players remains the same; they construct signature shares based on the
data to be signed. 

Because the behavior of honest players is the same as in Shoup's
scheme, we can focus on the ability of the malicious player to
construct a signature based on $\nums-1$ signature shares. As the
original scheme is secure and the malicious player receives strictly
less information in our scenario, we know that the adversary cannot
extract enough information from honest players' signature shares to
construct a \nums th signature share on his own. Because the signature
shares are RSA signatures, the shares are existentially unforgeable
and the adversary cannot create a valid threshold signature.

%We can even simulate the view of the malicious player in the
%same way as Shoup's original proof of security.

\subsection{Regarding Key Refresh}
\label{sec:proof-refresh}

\subsubsection{Key Refresh Algorithm}

Our code utilizes Shoup's threshold RSA signature
scheme~\cite{shoup-sig}. Thus, each player $i$ has a secret key share
$s_i$ defined as follows according to a polynomial secret sharing
scheme, where $d,m$ is the root secret key and $\leftarrow$ indicates
a uniform random choice of an element from a set:
\begin{eqnarray*}
a_0 & = & d \\
a_j & \leftarrow & \{ 0, \dots, m-1\} \\
f(x) & = & \sum_{j=0}^{\nums-1} a_j x^j \\
s_i & = & f(i)
\end{eqnarray*}

Note that we can define this related secret sharing scheme:
\begin{eqnarray*}
a_0 & = & d \\
a_j & \leftarrow & \{ 0, \dots, m-1\} \\
b_0 & = & 0 \\
b_j & \leftarrow & \{ 0, \dots, m-1\} \\
f_r(x) & = & \sum_{j=0}^{\nums-1} (a_j+b_j)x^j \\
{s_r}_i & = & f_r(i) \mod m
\end{eqnarray*}
This produces key shares for each player corresponding to the same
secret key as the original scheme, so the players can sign data under
the same public root key using shares from the old scheme or the new
scheme. (They cannot, however, mix shares created from different
polynomials.) We now construct a key refresh scheme based on this fact:
\begin{eqnarray*}
b_0 & = & 0 \\
b_j & \leftarrow & \{ 0, \dots, m-1\} \\
f_b(x) & = & \sum_{j=0}^{\nums-1} b_j x^j \\
{s_b}_i & = & f_b(i) \mod m \\
s_i' & = & s_i + {s_b}_i \mod m \\
     & = & \sum_{j=0}^{\nums-1} (a_j+b_j)x^j
\end{eqnarray*}

Given \nums\ key shares, computer code run by the players can run
polynomial interpolation code to reconstruct $m$. The players thus construct a new random polynomial $f_b$ such that
$f_b(0)=0$, giving each player $i$ ($1\leq i \leq \nump$) a new key share 
$s_i'$. The players begin to use their new key shares to sign 
data at the time decided upon by the root collective. In this way, the
players can refresh their secret key shares without changing the
public key.



\subsubsection{Security}

Next, we prove that our scheme for key refresh is secure.
\ignore{
\subsubsection{Refreshing without use of the secret key}

Coefficients for the polynomial secret sharing of $0$ which we utilize
for refreshing the players' secret keys should, ideally, be chosen
from $Z_m=\{ 0,\dots,m-1\}$. However, by choosing these coefficients
from $Z_n =\{ 0,\dots, n=1\}$, the players can refresh their key
shares without direct use of these keys.

To prove this remains secure, we prove that the probability that any
coefficient chosen during the key refresh program is, with
overwhelming probability, uniformly distributed over $Z_m$.

First, note that $n$ is a product of {\it safe primes}, chosen in the
following way:
\begin{eqnarray*} %1
p & = & 2p'+1 \\
q & = & 2q'+1 \\
n & = & pq
\end{eqnarray*}
If we were to choose our random coefficients from $Z_{m}$, these
coefficients would be uniformly distributed over $Z_m$.  We may then
divide the domain $Z_{\frac{n}{4}}$ into two sections: the part that
is of size $Z_{m}$ and the remainder. If no random coefficients are
chosen in this remainder, then the coefficients are uniformly
distributed over $Z_m$ as we require. However, it is sufficient to
prove that coefficients are chosen from this remainder with only
negligible probability; as there are only a polynomial number of
coefficients chosen, with overwhelming probability there will thus be
none in the remainder.

By the division theorem, there exist unique positive integers $c,r$
such that $n = cm + r$. We now prove that $c=4$ for $p',q' > 6$. As
$\frac{1}{p'}$ is chosen to be negligible, $p'$ must be larger than
$6$ in any case, even to be secure against a human attacker.
\begin{eqnarray*} %2
n & = & (2p'+1)(2q'+1) \\
m & = & p'q' \\
\lfloor \frac{4p'q' +
2p'+2q'+1}{p'q'} \rfloor & = & 4 + \lfloor \frac{2p'+2q'+1}{p'q'}\rfloor
\end{eqnarray*}
If $p'q' > 6$, then
\begin{eqnarray*} %3
\frac{1}{p'} & < & \frac{1}{6} \\
\frac{1}{q'} & < & \frac{1}{6} \\
\frac{1}{q'}+\frac{1}{p'} & < & \frac{1}{3} \\
p'q' & > & 3p' + 3q'
\end{eqnarray*}
We may also note that $p' + q' > 1$, so $p'q' > 2p'+2q' + 1$. Thus, 
\begin{eqnarray*} %4
\lfloor \frac{4p'q' + 2p'+2q'+1}{p'q'} \rfloor & = & 4 + \lfloor 
          \frac{2p'+2q'+1}{p'q'} \rfloor \\
    & = & 4
\end{eqnarray*}
We can thus derive the remainder, $r$, as $n-4m = 2p'+2q'+1$. Let the
probability that $a\in Z_{n}\backslash Z_{4m}$ for $a \leftarrow Z_n$
be denoted $y$. The probability that $a\in a\in
Z_{{\frac{n}{4}}}\backslash Z_{m}$ is $4y$.
\begin{eqnarray*} %5
y & = & \frac{r}{n} \\
\frac{1}{y} & = & \frac{4p'q'+2p'+2q'+1}{2p'+2q'+1} \\
    & = & \frac{4p'q'}{2p'+2q'+1} + 1 \\
\frac{1-y}{y} & = & \frac{4p'q'}{2p'+2q'+1} \\
\frac{y}{1-y} & = & \frac{2p'+2q'+1}{4p'q'} \\
    & = & \frac{1}{2q'} + \frac{1}{p'} + \frac{1}{4p'q'}
\end{eqnarray*}
As $p',q'$ are chosen to be large, $\frac{1}{2q'} + \frac{1}{p'} +
\frac{1}{4p'q'}$ is negligibly small. For clarity, we will represent
this probability as $\mbox{negl}$ for the remainder of this proof.
\begin{eqnarray*} %6
\frac{y}{1-y} & = & \mbox{negl} \\
\frac{1-y}{y} & = & \frac{1}{\mbox{negl}} \\
\frac{1}{y} & = & \frac{1+\mbox{negl}}{\mbox{negl}} \\
y & = & \frac{\mbox{negl}}{1 + \mbox{negl}}
\end{eqnarray*}
Note that $\frac{\mbox{negl}}{1 + \mbox{negl}}$ is also a negligible
probability, so $y$ and $4y$ are negligible. As $4y$ is the probability of
picking a coefficient from the `bad' set of remainders and there are
only polynomially many such coefficients, we can conclude that the
probability of picking any coefficient from the remainders is
negligible.

The secret sharing scheme remains secure because $s_i+s_i'$ remains
uniformly distributed modulo $m$ (the form of the secret key that
appears in the signature share exponent). Because the distribution of
this key does not change, an adversary cannot extract information from
it that it could not in the scheme without key refreshing.
}

%\subsubsection{General key refresh scheme}

Our key refresh scheme of Section~\ref{sec:refresh} essentially
requires that the players construct polynomial shares of $0$ and add
them to their original secret shares of the secret key to generate new
shares of the secret key. The security of this scheme for
uncompromised players follows directly from the security of the
polynomial secret sharing scheme~\cite{shamirshare}. If a player's
original secret share remains secret, adding another secret share to
the original share does not let the adversary reconstruct it.

We must also ensure that $t$ or fewer players whose shares,
$s_{i_1},\dots,s_{i_t}$, are known to the adversary are also protected
by this scheme. Because some players that were previously corrupted
may remain corrupted and some new players may become corrupted during
the key refresh process, we allow the adversary to learn at most $t$
shares of the refresh key, $s_{j_1}',\dots, s_{j_t}'$.

We must now show by reduction that a player $i$ ($i \in
\{i_1,\dots,i_t\}, i\not\in \{j_1,\dots,j_t\}$, whose secret share is
known to the adversary but whose refresh share is not, gains a secret
key share through the key refresh process. Note that if the adversary
could reconstruct $s_i+s_i'$ with non-negligible probability, the
adversary could also reconstruct $s_i'$ with non-negligible
probability, as he knows $s_i$. Given only $t$ or fewer shares of the
refresh key and $t$ or fewer shares of the original key, the adversary
cannot reconstruct $s_i'$ with non-negligible probability. Thus, the
adversary cannot reconstruct player $i$s new key share $s_i+s_i'$ with
non-negligible probability.
