\section{How to Sign Data}
\label{sec:how-sign}

After the players have generated keys as described in
Section~\ref{sec:keygen}, any \nums\ players can jointly sign data.

\paragraph{Creating signature shares.}
If one player wishes to sign some data, such as a certificate, he
sends it to all other players and requests that they create signature shares for this data. 
(Note that the requesting player needs to receive at least $k-1$ correct responses.) Those players then check the data according to their established security routines\footnote{For example, the players might institute a
rule that they will only sign X.509 certificates that delegate previously unallocated chunks of address space to one RIR, or a rule that they will only sign certificates vouched for by IANA.}.

% TODO -- other security checks?

If a player wishes to help create a signature on the data, that player constructs a signature
share. In our code, we utilize Shoup's threshold RSA scheme, so player
$i$ ($1 \leq i \leq \nump$) with secret key $s_i$ signs data $x$ to
produce signature share $x_i$ as follows:
\begin{eqnarray*}
x_n & = & x \mod n \\
x_i & = & x_n^{2 s_i \nump!}
\end{eqnarray*}
Note that this is a standard RSA signature calculation with secret key
$2s_i \nump!$. If the player is utilizing a cryptographic signature
box as described in Section~\ref{sec:use-box}, then he simply requests
that the box produce a signature on data $x$ with key $2 s_i \nump!$ to
produce the signature share $x_i$.

\paragraph{Finding corrupted players.}
The root collective can determine which player(s) are corrupted or
malicious by combining signature shares as we discuss in this section;
if some players' shares combine into a valid signature, the players who
created those shares created them correctly.

If all five participants respond then there are ten possible
combinations of the private key shares. If there is exactly one corrupt
player then one can create four valid combinations and six invalid
combinations. Determining the corrupt player involves identifying the
member absent from the valid combinations and verifying that the
suspect player player is common within all invalid combinations.  In
the case of two corrupt players, where all participants have
responded, there will be only one valid combination and nine invalid
combinations. The players not involved in the valid combination are
identifiable as the corrupted players.

We encourage all players to participate and
contribute their signature shares even if \nums\ participants have
already responded. If all players participate, it becomes possible to
identify problems or corrupt key shares at an earlier time.

In the full Shoup scheme~\cite{shoup-sig}, each player generates a small non-interactive zero-knowledge proof of
correctness along with their signature share. These shares do not
divulge secret information; they only allow any player to efficiently
check that signature shares were constructed correctly. However, in this application, these proofs are not necessary and interfere with the use of hardware security modules. We discuss the
security of Shoup's signature scheme without proofs of correctness in
Appendix~\ref{sec:proof-sigshare}.


\ignore{
The players may employ proofs of correctness in their generation of
signature shares. In this, the full Shoup scheme~\cite{shoup-sig}, each
player generates a small non-interactive zero-knowledge proof of
correctness along with their signature share. These shares do not
divulge secret information; they only allow any player to efficiently
check that signature shares were constructed correctly. We discuss the
security of Shoup's signature scheme without proofs of correctness in
Appendix~\ref{sec:proof-sigshare}.

Before combining signature shares, the player combining these shares
verifies their attached proofs. If a proof is incorrect, then the
combining player alerts the root collective that the player who sent
it may be corrupted. The combining player now has fewer than \nums\
valid signature shares, so he must request additional shares from root
collective players in order to construct a valid signature.
}

\paragraph{Combining signature shares.}
Once \nums\ players have produced signature shares, each player sends
their share to the requesting player. Once he has received these
shares, the requesting player combines the shares into a joint
signature and then checks that the signature is valid under the root
public key. If it is not valid, then one of the signature shares is
invalid; the player that produced it may be corrupt, incorrect, or an adversary may have interfered with the transmission.
%(If the players are utilizing proofs of correctness, this is
%not a concern.)
The combining player should combine other sets of signature shares, as we describe above, to discover
which player (or players) sent an invalid share. Generally, the
combination procedure works as follows:
\begin{itemize}
\item[{\it Input}] Root public key $(e,n)$, signature
shares $x_{i_1},\dots,x_{i_\nums}$ from distinct players
$i_1,\dots,i_\nums$, respectively, data to be signed $x$.

\item[{\it Output}] Signature of $x$ under root public key $(e,n)$.
\end{itemize}
\ignore{
\begin{eqnarray*}
S & = & \{ i_1,\dots, i_\nums\}\\
\Delta & = & \nump !\\
\lambda_{c_1,c_2}^S & = & \Delta \frac{\prod_{c'\in S\backslash
                          \{c_2\}} (c_1-c')}{\prod_{c'\in S\backslash
                          \{c_2\}} (c_2-c')} \mod n \\ 
w & = & x_{i_1}^{2\lambda_{0,i_1}^S}\cdots 
        x_{i_\nums}^{2\lambda_{0,i_\nums}^S} \mod n \\
e' & = & 4\Delta^2
\end{eqnarray*}
Using the extended Euclidean algorithm, calculate integers $a,b$ such
that $ae' +be = 1$. The combined signature is $w^ax^b \mod n$.  }
Essentially, the exponents of the signature shares are points from a
polynomial that encodes the root secret key. Using polynomial
interpolation, we can combine these shares in such a way that the
exponent becomes the signing key. In this way, we produce a signature
without revealing the secret key. Cryptographic details and
mathematical intuition are discussed in Shoup's
paper~\cite{shoup-sig}.

