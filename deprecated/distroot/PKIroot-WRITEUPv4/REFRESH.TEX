\section{Issues Surrounding Key Refresh}
\label{sec:refresh}

Should it become known that one or more of the key shares (but less
than the threshold required to perform signing) have become
compromised, key refresh should be initiated. One must also take into
account the fact that silent key share compromise might have occurred
creating an unknown or undisclosed compromise scenario. To help
prevent against both known and unknown compromise we recommend that
the root collective refresh their key shares periodically in addition
to mandatory refresh events due to known compromise.

One possible key refresh schedule is three to four times per year at
the regular meetings of the RIR representatives. Key refresh creates
new signing key shares for each player that create signatures valid
under the same public key. While they could instead generate shares
corresponding a new public key each time, the logistical difficulties
involved in distributing the new public key to every single end user
makes this approach difficult. By refreshing their key shares, they
can protect against an adversary silently
 gaining access to fewer
than \nums\ shares of the secret key; the adversary must collect last
least \nums\ key shares that were created together to forge
signatures. Note that, because the public key does not change, attacks
based only on knowledge of the public key, such as brute force and
factoring, can proceed independent of key refreshing. The root
collective must thus change their public key periodically as well, but
at much greater intervals.

First, the root collective generates new shares of the secret signing
key. Then, they switch to signing with the new shares at an
agreed-upon time.  It is imperative that each participant destroy
their old key shares as soon as possible, as their continued existence
creates extra risk. Should an attacker gain access to three key shares
created at the same time, the attacker could successfully sign under
the existing public key.


%In this section we will present two different key refresh
%scenarios, along with recommended procedures for each one.

\ignore{
\subsection{Physical presence, no key required.}
All \nump\ players physically gather to refresh
their keys. Procedural security should be as described in
Section~\ref{sec:keygen}. Instead of running key generation software,
however, the root collective should run key refreshing software. (We
describe our software for key refresh, designed for this scenario, in
Section~\ref{sec:code}.) Note that, because the players' secret keys
are not required for this scenario, they do not need to add security
to transport them to the key refresh meeting.

Our code utilizes Shoup's threshold RSA signature
scheme~\cite{shoup-sig}. Thus, each player $i$ has a secret key share
$s_i$ defined as follows according to a polynomial secret sharing
scheme, where $d,m$ is the root secret key and $\leftarrow$ indicates
a uniform random choice of an element from a set:
\begin{eqnarray*}
a_0 & = & d \\
a_j & \leftarrow & \{ 0, \dots, m-1\} \\
f(x) & = & \sum_{j=0}^{\nums-1} a_j x^j \\
s_i & = & f(i)
\end{eqnarray*}

Note that we can define this related secret sharing scheme:
\begin{eqnarray*}
a_0 & = & d \\
a_j & \leftarrow & \{ 0, \dots, m-1\} \\
b_0 & = & 0 \\
b_j & \leftarrow & \{ 0, \dots, \frac{n}{4}-1\} \\
f_r(x) & = & \sum_{j=0}^{\nums-1} (a_j+b_j)x^j \\
{s_r}_i & = & f_r(i)
\end{eqnarray*}
This produces key shares for each player corresponding to the same
secret key as the original scheme, so the players can sign data under
the same public root key using shares from the old scheme or the new
scheme. (They cannot, however, mix shares created from different
polynomials.) We now construct a key refresh scheme based on this fact:
\begin{eqnarray*}
b_0 & = & 0 \\
b_j & \leftarrow & \{ 0, \dots, \frac{n}{4}-1\} \\
f_b(x) & = & \sum_{j=0}^{\nums-1} b_j x^j \\
{s_b}_i & = & f_b(i) \\
s_i' & = & s_i + {s_b}_i \\
     & = & \sum_{j=0}^{\nums-1} \left(a_j+\right)x^j
\end{eqnarray*}
The players thus construct a new random polynomial $f_b$ such that
$f_b(0)=0$, giving each player $i$ ($1\leq i \leq \nump$) a refresh
share ${s_b}_i$. Each player keeps this secret and transports it
home. If no player's key share has been compromised, he then creates
his new key share $s_i' = s_i + {s_b}_i$ and starts using it to sign
data at the time decided upon by the root collective. In this way, the
players can refresh their secret key shares without changing the
public key. We show the security of this key refresh scheme in
Appendix~\ref{sec:proof-refresh}.

While this scheme has the advantage of reducing the risks of key
transport, it does increase the size of each player's key share. After
the players have refreshed their keys $r$ times, their shares are of
size at most $1 + \lg m + (\nums -1)\lg \nump + (\nums-2) + r$, which
for $\nump=5, \nums=3$ is an increase of at most $8+r$ bits.
}

%\subsection{Physical presence}
%\label{sec:refresh-mod}

%In some scenarios, the players cannot utilize our previous scheme
%because their cryptographic hardware cannot use slightly larger secret
%keys. If this is the case, 
The players physically gather and bring their secret keys
 to the key
refresh meeting. Transporting these keys requires physical
 and
computational security measures similar to the key-transport
protocols we describe in Section~\ref{sec:keygen}.

The trusted official sends out a public key to all players; only the
official knows the corresponding secret key. Each player encrypts
their key share $s_i$ using this public key and loads it onto
removable media, along with the public key required by the share
generation procedure.
 
 They then run key refresh software in the
same
 manner as key generation software was run in
Section~\ref{sec:keygen}. The trusted official loads the encrypted key
shares into the computer to be utilized for the key refresh procedure
and decrypts them. The official then utilizes the key refresh code to
ensure that every key share is correct: if $k$ key shares combine to
create a valid private key corresponding to the root public key, those
key shares are correct. The trusted official then generates the new
key shares using the key refresh code, destroys the old key shares,
and distributes the new key shares as in Section~\ref{sec:keygen}.  We
discuss a method for key refresh in Appendix~\ref{sec:proof-refresh}.



%We do not provide code for this method of key refreshing; instead we
%discuss the differences between this method and the less complex one
%above.


\ignore{

Our code utilizes Shoup's threshold RSA signature
scheme~\cite{shoup-sig}. Thus, each player $i$ has a secret key share
$s_i$ defined as follows according to a polynomial secret sharing
scheme, where $d,m$ is the root secret key and $\leftarrow$ indicates
a uniform random choice of an element from a set:
\begin{eqnarray*}
a_0 & = & d \\
a_j & \leftarrow & \{ 0, \dots, m-1\} \\
f(x) & = & \sum_{j=0}^{\nums-1} a_j x^j \\
s_i & = & f(i)
\end{eqnarray*}

Note that we can define this related secret sharing scheme:
\begin{eqnarray*}
a_0 & = & d \\
a_j & \leftarrow & \{ 0, \dots, m-1\} \\
b_0 & = & 0 \\
b_j & \leftarrow & \{ 0, \dots, m-1\} \\
f_r(x) & = & \sum_{j=0}^{\nums-1} (a_j+b_j)x^j \\
{s_r}_i & = & f_r(i) \mod m
\end{eqnarray*}
This produces key shares for each player corresponding to the same
secret key as the original scheme, so the players can sign data under
the same public root key using shares from the old scheme or the new
scheme. (They cannot, however, mix shares created from different
polynomials.) We now construct a key refresh scheme based on this fact:
\begin{eqnarray*}
b_0 & = & 0 \\
b_j & \leftarrow & \{ 0, \dots, m-1\} \\
f_b(x) & = & \sum_{j=0}^{\nums-1} b_j x^j \\
{s_b}_i & = & f_b(i) \mod m \\
s_i' & = & s_i + {s_b}_i \mod m \\
     & = & \sum_{j=0}^{\nums-1} (a_j+b_j)x^j
\end{eqnarray*}
Given \nums\ key shares, computer code run by the players can run
polynomial interpolation code to reconstruct $m$. The players thus construct a new random polynomial $f_b$ such that
$f_b(0)=0$, giving each player $i$ ($1\leq i \leq \nump$) a new key share 
$s_i'$. The players begin to use their new key shares to sign 
data at the time decided upon by the root collective. In this way, the
players can refresh their secret key shares without changing the
public key. We show the security of this key refresh scheme in
Appendix~\ref{sec:proof-refresh}.


\subsection{No physical presence}

If the root collective cannot physically gather to refresh their key
shares, they can securely perform a modified version of the scheme we
present above over a network connection. This setup has the advantage that the key shares do not have to physically travel to a key refresh meeting. Unfortunately, the players do have to put their key shares on a network-attached computer, which carries risks of its own.

We have not written code for this key refresh scheme, but it can be written as general cryptographically multi-party computation~\cite{gbw}. Each step of the following algorithm is performed {\it obliviously} by the participants; they perform a computation without learning the intermediate or final results beyond their resulting key shares. Each message in the
protocol must be signed with that player's secret key share and
verified by the recipients.

The players begin by reconstructing $m$ as we outline above. They then
jointly choose random coefficients for the polynomial sharing scheme;
player $i$ ($1 \leq i \leq \nump$) chooses $b_{i,1},\dots ,
b_{i,\nums-1}$.
\begin{eqnarray*}
b_0 & = & 0 \\
b_{i,j} & \leftarrow & \{ 0, \dots, m-1\} \\
f_b(x) & = & \sum_{j=0}^{\nums-1} \left(\sum_{i=1}^\nump b_{i,j}\right) x^j \\
{s_b}_i & = & f_b(i) \mod m \\
s_i' & = & s_i + {s_b}_i \mod m \\
     & = & \sum_{j=0}^{\nums-1} \left(a_j+\left(\sum_{i=1}^\nump b_{i,j}\right)\right)x^j \mod m
\end{eqnarray*}

Because $m$ and the coefficients $\sum_{i=1}^\nump b_{i,1}, \dots ,
\sum_{i=1}^\nump b_{i,\nums-1}$ cannot be revealed, the players must
cryptographically secure this computation. Each player $i$ ($1 \leq i
\leq \nump$) must learn $s_i'$ and no other information. 
}
