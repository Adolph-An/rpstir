\section{How to Generate Keys}
\label{sec:keygen}

The first step in utilizing threshold signatures is to generate the
root public key and corresponding private key shares. To ensure the
security and integrity of these key shares, all players of the root
collective will physically gather for the key generation ceremony.  We
propose the security guidelines outlined in this section to prevent
the key generation process from leaking private information or being
biased in favour of any player.

\subsection{Key Generation}

Representatives of the RIRs meet in person several times per
year. These conferences afford the opportunity for `out-of-band'
creation of the root public key and private key shares.  Because the
RIR representatives have previously met face-to-face, they can avoid
the complex issues of remote identity verification by performing key
generation during one of these gatherings. In addition, regular physical
meetings provide an attractive setting for periodic key refresh, as
outlined in Section~\ref{sec:refresh}.

To encourage equality and fair play, the root collective first must
agree upon a trusted ceremony official. The trusted official will
operate the software and hardware involved in the key generation
ceremony. Prior to the key generation ceremony, each RIR also creates a
public/private RSA key pair exclusively for protecting their private
key share during the ceremony.

During the ceremony, the trusted official presents the hardware and
software to the RIR representatives for inspection. The official then
generates the root public key and private key shares. To give each
representative their RIR's key share, the official collects each RIR's
public key on removable media, uses each public key to encrypt one key
share, and gives each representative the corresponding encrypted key
share and root public key on removable media. We recommend the use of
encryption to protect the key shares; this prevents participants from
obtaining access to other players' key shares. The trusted official
then wipes all data from the key generation hardware. We defer details
of this procedure to Appendix~\ref{sec:keygen-detail}.

Before concluding the key generation ceremony, the players may wish to
verify that the collective can correctly sign data. The players then
protect their key shares through the use of cryptographic and physical
security before transporting them to their respective organizations.


In our distributed signature scheme, we utilize the Shoup threshold
RSA signature
 scheme~\cite{shoup-sig}. Key generation for this
scheme operates
 generally as follows:
\begin{itemize}
\item[{\it Input}] Number of players \nump, number of players required
to create a signature \nums, key length. (Key length should be chosen
to create a secure RSA signature; we recommend at least 4096 bits.)

\item[{\it Output}] Root public key $(e,n)$, secret key share $s_i$
for each player $i$ ($1 \leq i \leq \nump$). (In the full scheme, which we do not utilize, players may also receive verification keys $v,
v_1, \dots, v_\nump$. For discussion of proofs of verification for
signature shares, see Section~\ref{sec:how-sign}.)
\end{itemize}
Any \nums\ of the signature shares together define a polynomial that
encodes the root secret key; given \nums\ of the shares we can
reconstruct this secret key (though we do not need to do so to produce
signatures)~\cite{shamirshare}.

\ignore{
We thus perform key generation as follows for
\nump\ players, where \nums\ are required to sign data:
\begin{enumerate}
\item Generate two large primes, $p,q$ such that for primes $p',q'$,
$p=2p'+1$ and $q=2q'+1$. Let $n=pq$ and $m=p'q'$. Choose $e$ to be a
prime such that $\nump < e < m$. The root public key is $(e,n)$, and
is given to all players.
\item Calculate $d$ such that $de \equiv 1\ (\mbox{mod } m)$. For each
player $i$ ($1\leq i\leq n$), calculate a secret key share $s_i$ as
follows:
  \begin{eqnarray*}
    a_0 & = & d \\
    a_j & \leftarrow & \{ 0 ,\dots, m-1\} \\
    f(x) & = & \sum_{j=0}^{\nums-1} a_j x^j \\
    s_i & = & f(i)
  \end{eqnarray*}
Each player $i$ ($1 \leq i \leq \nump$) receives only their own secret
key share $s_i$.

\item If the root collective anticipates that they might utilize
proofs of signature share correctness (see Section~\ref{sec:how-sign}),
then choose $v$ randomly from $Q_n$ (the subgroup of squares in
$Z_n^*$) and calculate $v_i = v^s_i \mod n$ ($1 \leq i \leq
\nump$). All players receive $v,v_1,\dots,v_\nump$.
\end{enumerate} 
}


\subsection{Key Transport}

Each player must protect the secrecy and integrity of their private
key share during transport. It is extremely important that the private
key shares not be disclosed either accidentally or to malicious
adversaries actively attacking the protection measures. Thus, we
recommend that the key shares remain encrypted (and separated from the
decryption key) during transit. In addition, the key share should not
reside on systems connected to public networks. Participants may wish
to incorporate physical security measures, including hardware
cryptographic devices and physical locks.

If a player's key is potentially compromised, he must inform the root
collective of this fact. If \nums\ key shares are compromised, the
root key {\it must} be immediately retired and not used further. If
fewer than \nums\ keys are compromised, then the root collective can
refresh their key shares as soon as possible and make a decision as to
how soon they will retire the root key.

By refreshing their key shares, the root collective creates new key
shares to all players without changing the root public key. Because of
the logistical difficulties entailed in distributing a root key, the
root collective should perform key refresh regularly to guard against
known and unknown key compromises.
