\section{Code Summary}
\label{sec:code}

In this section, we give an overview of Java code~\cite{sig-code} for
a distributed root based on Shoup's threshold RSA signature
scheme~\cite{shoup-sig}.

\subsection{Key Generation}

The root collective players run the following code on a secure
computer to generate key shares for each player.
\begin{enumerate}
\item Create a new {\tt Dealer} object to create keys of size {\tt keysize}.

        {\tt d = new Dealer(keysize)}

\item	Use the Dealer object to generate key shares for \nump\
	players such that at least \nums\ players must collaborate to
	successfully sign data under the root key.

 	{\tt d.generateKeys(\nums, \nump)} 

\item	Retrieve the root verification key from the {\tt Dealer}
	object. This key can be utilized to verify all valid
	signatures constructed by the root collective; it is utilized
	just as any other RSA public key.

	{\tt gk = d.getGroupKey()} 

\item	Retrieve the key shares from the {\tt Dealer} object. Each key
	should be written to a separate piece of media, which is given
	to the corresponding player.

	{\tt KeyShare[] keys = d.getShares()}
\end{enumerate}

\ignore{
\subsection{Key Refresh}

Periodically, the root collective should {\it refresh} their key shares,
using  the method discussed in Section~\ref{sec:refresh}. This protects against compromise (silent or known) of at most $k-1$ key shares.

We discuss code for one method here.
}
%FINISHME

\subsection{Constructing Signature Shares}

%If the players are utilizing the Shoup scheme without proofs (which we
%discuss in Section~\ref{sec:how-sign}), then each player {\it i} can
%generate their signature share as in the full scheme. Alternatively,
Given their key share {\tt ei} and root key modulus {\tt n}, each player can
calculate their signature share as in a standard RSA signature:\\
{\tt BigInteger bmod = new BigInteger(b).mod(n)} \\
{\tt BigInteger sigshare = bmod.modPow(ei, n)}

Each of the signing players send {\tt sigshare} and their identifier $i$ to
the party combining signature shares into a valid root signature.

Note that our code implements the full Shoup scheme, including non-interactive zero-knowledge proofs. If the root collective were utilizing the full Shoup
scheme~\cite{shoup-sig}, then each signing player would utilize his
KeyShare object {\tt ks} to create a signature share (with associated proof
of correctness) on the data {\tt b} as follows: {\tt ks.sign(b)}.

\subsection{Combining Signature Shares}

Given the data to be signed {\tt b}, an array of \nums\ valid
signature shares {\tt sigs}, the number of signers \nums\ and total
number of players \nump, as well as the root public key object {\tt
gk}, one combines the signature shares into a root signature on {\tt
b} as follows: {\tt
SigShare.combine(b,sigsq,k,l,gk.getModulus(),gk.getExponent())}.

%\subsection{Verifying Signature Shares}

If the root collective were utilizing the full version of Shoup's
threshold signature scheme~\cite{shoup-sig}, they would have to verify the
validity of each share before combining them into a full root
signature. Given the data to be signed {\tt b}, an array of \nums\
signature shares {\tt sigs}, the number of signers \nums\ and total
number of players \nump, as well as the root public key object {\tt
gk}, one verifies the validity of the signatures as follows: {\tt
SigShare.verify(b, sigs, k, l, gk.getModulus(), gk.getExponent())}.


\subsection{Verifying Joint Signature}

Given the data that was to be signed {\tt b}, the signature to be checked {\tt sig}, and the root public key {\tt e,n}, one may verify the validity of the signature as with any RSA signature:
{\tt sig.modPow(e,n).compareTo(b) == 0}.
