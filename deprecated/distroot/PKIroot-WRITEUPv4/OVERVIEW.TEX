\section{Overview and Procedures}
\label{sec:overview}


In order to operate a distributed signing root based on threshold
signatures, the \nump\ players in the root collective must first
determine \nums, the number of players required to create a
signature. This must be chosen such that the maximum number of
malicious or compromised players always be less then \nums. (For our
proposal, we recommend $\ell=5$, $k=3$.) They then gather to create a
common public key; each player gets one share of the secret key. In
Section~\ref{sec:keygen}, we describe this process, as well as the
physical and computational security precautions to be taken.

Each player returns to their operations center with a share of the
root signing key on removable media, which they load into the
computational device that will construct signature shares. We discuss
some of the issues involved in utilizing secured cryptographic signing
boxes for this purpose in Section~\ref{sec:use-box}.

Any \nums\ players of the root collective can now collaborate to sign
data such that it can be verified with the root public key. When one
player decides to sign some data, that player sends it to all other
players. Each of these players examines the data for correct structure
and content (e.g., that the address space being allocated is being
correctly assigned). Each signing player then uses their secret share
of the root signing key to produce a signature share, then sends this
share to the requesting player. Once the requesting player has
gathered valid signature shares from \nums\ distinct players, the
requesting player combines them into a signature valid under the root
verification key. We show how these signature shares are constructed
and combined in Section~\ref{sec:how-sign}, as well as discuss
procedures for producing signatures and identifying malicious or
compromised players.
\ignore{
Key shares can be silently compromised by malicious entities who learn
secret signing key shares. To guard against this, the root collective
should periodically refresh their key shares. This process allows them
to acquire new key shares without changing the public key. As securely
distributing a new root verification key for a PKI is so difficult,
this approach improves both security and practicality. We discuss an
approach to refreshing key shares in Section~\ref{sec:refresh}.
}

Private key shares should be well protected. However, through
administrative error or other means it is possible that a private
key share might become known to a malicious entity. To protect against
such an entity from obtaining enough private key shares that they can
sign data as the root of the PKI, we recommend that the root
collective periodically refresh their private key shares. A refresh of
private key shares produces new key shares for each one of the
participants without requiring a new public key to be issued. This
removes the overhead and complexity involved in introducing a new root
key unrelated to previously-issued certificates while maintaining
security in regards to known and unknown key share compromise. We
discuss several approaches to refreshing key shares in
Section~\ref{sec:refresh}.

\ignore{
In Section~\ref{sec:code}, we describe code for operating a
distributed signature root collective based on Shoup's threshold RSA
signature scheme~\cite{shoup-sig}. This includes code for key
generation, producing signature shares (including all the options we
discuss in Section~\ref{sec:how-sign}), combining these signature
shares into a threshold signature, and refreshing each players' secret
key.
}