\section{Concept and Operation of a Distributed Root}
\label{sec:operation}

The root of a PKI signs certificates for its subordinates; in turn,
these subordinates may sign certificates for subordinates of their
own\footnote{Subordinates may issues certificates only if they are
permitted to do so by being specified as a certificate authority in
their X509 certificate.}. The root public key acts as the ultimate
trust anchor within the proposed PKI. As such, it is an extremely
tempting target for attack. In a scenario where multiple parties
represent the public key, the security of the root private key can be
increased by employing a threshold signature scheme to split the root
private key among several players. Quorums of root collective members
distributed over the globe collectively act as the root signing entity
by jointly signing data to be verified with a single root verification
key. Thus the RIRs are responsible for managing address allocations
and certificates without requiring IANA to engage in such day-to-day
operational issues.


All \nump\ participants in the root collective jointly generate key
shares for
 each player such that $\nums\leq \nump$ players must
collaborate to
 sign data; no group of fewer than $\nums$ players can
sign data. (See Section~\ref{sec:overview} for details.) At any later
time they can sign data as
 follows: any \nums\ players individually
construct shares of the
 signature using their key share; any
computer may then combine these
 signature shares into a valid
signature under the root key.

Splitting the root signing key in this way protects against up to
$\nums-1$ players being compromised or attempting to cheat. In
combination with strong physical security, the root collective can
achieve robust protection of the root signing key while simultaneously
distributing control among the participants. To protect against
undetected compromise of root collective participants, we recommend
they refresh their key shares as we discuss in
Section~\ref{sec:refresh}. Use of threshold signature can also
increase availability, as $\nump-\nums$ players may be unavailable or
slow in response without hindering the signature process.

\ignore{To increase their ability to detect a cheating or corrupted player of
the root collective, the players may employ proofs of
correctness. Along with his signature share, each player also
constructs a small proof that his share was correctly
constructed. Some players, however, may wish to use cryptographic
hardware that cannot generate such proofs. The scheme remains secure
even in the absence of these proofs; it is only somewhat harder to detect a
cheating or corrupted member of the root collective.}


This paper is a companion to software (see Appendix~\ref{sec:code})
that performs key generation, signing, and verification according to
Victor Shoup's threshold RSA signature scheme~\cite{shoup-sig}. In
addition, we have added the capability to perform key refresh
as we outline in Appendix~\ref{sec:proof-refresh}. Operated according to
the procedures described in this document, this code performs the core
tasks necessary for a collective of players to operate a virtual
signing authority.
